%________________________________________________________________________
%------------------------------------------------------------------------
%							Font/Schriftarten Setup
%/\/\/\/\/\/\/\/\/\/\/\/\/\/\/\/\/\/\/\/\/\/\/\/\/\/\/\/\/\/\/\/\/\/\/\/\
%\usepackage[scaled]{uarial}
% \usepackage{bookman}
%\usepackage{textcomp}%
%------------------------------------------------
% \renewcommand*{\ttdefault}{lmtt}%
%<-------------------->
% Create a command to format as typewrite with enabled hyphenation
% (Silbentrennung)
% 	Usage: \texttthyph{}
%<-------------------->
\makeatletter
\DeclareRobustCommand\ttfamily
        {\not@math@alphabet\ttfamily\mathtt
         \fontfamily\ttdefault\selectfont\hyphenchar\font=-1\relax}
\makeatother
\DeclareTextFontCommand{\texttthyph}{\ttfamily\hyphenchar\font=45\relax}
% Standartmäßig verfügbare Schriftarten:
% cmtt 	Computer Modern Typewriter (default)
% lmtt 	Latin Modern
% pcr 	Courier
%/\/\/\/\/\/\/\/\/\/\/\/\/\/\/\/\/\/\/\/\/\/\/\/\/\/\/\/\/\/\/\/\/\/\/\/\
%							Font/Schriftarten done
%------------------------------------------------------------------------
%________________________________________________________________________
%
%
%
% 								%##########################
% 								% Old `fontenc'-Stuff, pre Lua(La)Tex
% 								%##########################
% 								%|=======================||||
% 								% \usepackage[T1]{fontenc}%|| Alt, pre-Lua(La)Tex
% 								%|=======================||||
% 								% % %\usepackage[scaled]{uarial}
% 								% % % \usepackage{bookman}
% 								%|===================||||
% 								% \usepackage{lmodern}%|| Lädt Latin Modern Font und setzt es fürs Dokument
% 								%|===================||||
% 								% % %Ist weniger Pixelig in pdf als Latex Standart
% 								%|====================||||
% 								% \usepackage{textcomp}%|| Lädt Text Companion Font
% 								%|====================||||
% 								% % %Stellt insbesondere Zeichen zur verfügung, wie
% 								% % %baht, bul­let, copy­right, mu­si­cal­note, onequar­ter, sec­tion, and yen
% 								% % %------------------------------------------------
% 								% % % Standartschriftart festlegen:
% 								% % % Mögliche Werte
% 								% % % \rmdefault - Roman (Serifen) Font
% 								% % % \sfdefault - Sans Serif Font
% 								% % % \ttdefault - TypeWriter Font
% 								%|=========================================||||
% 								% \renewcommand*{\familydefault}{\rmdefault}%||
% 								%|=========================================||||
% 								% % %Die drei Schriftfamilien einstellen:
% 								%|==============================||||
% 								% \renewcommand*{\rmdefault}{lmr}%||
% 								%|==============================||||
% 								% % % Standartmäßig verfügbare Schriftarten:
% 								% % % cmr 	Computer Modern Roman (default)
% 								% % % lmr 	Latin Modern Roman
% 								% % % pbk 	Bookman
% 								% % % bch 	Charter
% 								% % % pnc 	New Century Schoolbook
% 								% % % ppl 	Palatino
% 								% % % ptm 	Times
% 								%|===============================||||
% 								% \renewcommand*{\sfdefault}{lmss}%||
% 								%|===============================||||
% 								% % % Standartmäßig verfügbare Schriftarten:
% 								% % % cmss 	Computer Modern Sans Serif (default)
% 								% % % lmss 	Latin Modern Sans Serif
% 								% % % pag 	Avant Garde
% 								% % % phv 	Helvetica
% 								%|===============================||||
% 								% \renewcommand*{\ttdefault}{lmtt}%||
% 								%|===============================||||