\input{"./2ProjectSetup".tex}%
\input{"\DenKrLayoutMainRootDir/2includes/packages/preamble_pre".tex}%
\documentclass[tikz,fontsize=11pt,class=scrbook]{standalone}% I.e. the content from \input{"\DenKrLayoutMainRootDir/2layout/tikz_standalone/preamble_1_class".tex}%
\input{"\DenKrLayoutBaseRootDir/tikz_standalone/1TikzStandalonePicIncludeThis".tex}%
\DenKrTikzStandalonePre%
%
%
%
\newcommand\nodedistance{1.0cm}%
\newcommand\innabst{0.8em}% Abstand zwischen Text und Zellenrand
%
\let\breite\undefined%
\newlength\breite%
\settowidth{\breite}{\large \textbf{Interfacename}}%
\setlength\breite{\breite+\innabst+\innabst+1em}%
%
%
\begin{tikzpicture}[%
	scale = \tikzpicturescale,%
	auto,%
	node distance=\nodedistance%
]%
\newcommand{\cnt}{2}%
%
%
%
% \bfseries%
%
%\draw[color=slightblue,fill] (-0.5,-4.5) rectangle (10.5,0.5);
%
%
\tikzstyle{legendnote}=[%
	cleartext,%
	align=left,%
	text width=45em,%
	anchor=north west,%
	xshift=0cm%
]%
%
%
%
%
	\node(start)%
		[startstop]%
		{Start};%
	\draw[arrow](start.south)--+(0,-0.5cm);
	\node(stop)%
		[startstop,below=0.8cm of start,xshift=2cm]%
		{Ende};%
	\draw[arrow,<-](stop.north)--+(0,0.5cm);
\node(textleftanchor)[coord,xshift=1.25cm]%
	at(start.north east -|stop.north east){};%
%
	\node(startnote)%
		[legendnote]%
		at(start.north east-|textleftanchor)%
		{Spezifiziert Anfang oder Ende eines Algorithmus. Die Bearbeitung beginnt
		am Knoten >>Start<< und endet bei Erreichen des Knotens >>Ende<<.};%
%
%
%
%
%
%
%
%
%
	\node(process)%
		[process,below=of stop,xshift=-1.0cm,yshift=-0.5cm]%
		{Prozess};%
	\draw[arrow](process.south)--+(0,-0.5cm);
	\draw[arrow,<-](process.north)--+(0,0.5cm);
%
	\node(processnote)%
		[legendnote,yshift=0.5cm]%
		at(process.north east-|textleftanchor)%
		{Prozess Knoten. Enthält eine Verarbeitung, einen konkreten Vorgang
		innerhalb des Algorithmus. Kann einzelne Operationen oder größere Teile des
		Algorithmus darstellen; möglicherweise exakte Code-Fragmente eines
		Programms, ggf. in Pseudo-Code.};%
%
%
%
%
%
%
%
%
%
	\node(decision)%
		[decision,anchor=north,yshift=-\nodedistance-0.5cm]%
		at($(process.south)+(0,-0.5cm)$)%
		{Entscheidung};%
	\node(decision2)%
		[decision2,below=1.0cm of decision]%
		{Entscheidung};%
	\draw[arrow,<-](decision.north)--+(0,0.5cm);
	\draw[arrow,<-](decision2.north)--+(0,0.5cm);
	\draw[arrow](decision.south east)-|%
		node[pos=0.6,anchor=west]{Ja}%
		+(0.5cm,-0.5cm);
	\draw[arrow](decision.south west)-|%
		node[pos=0.6,anchor=east]{Nein}%
		+(-0.5cm,-0.5cm);
	\draw[arrow](decision2.east)-|%
		node[pos=0.6,anchor=west]{Ja}%
		+(0.5cm,-0.5cm);
	\draw[arrow](decision2.west)-|%
		node[pos=0.6,anchor=east]{Nein}%
		+(-0.5cm,-0.5cm);
%
	\node(decisionnote)%
		[legendnote,yshift=0.5cm]%
		at(decision.north-|textleftanchor)%
		{Ja-Nein-Entscheidung. Der Knoten trifft eine spezifizierte Entscheidung.
		Je nach Resultat wird einer der beiden herausführenden Wege für den
		weiteren Verlauf des Algorithmus verfolgt.\\%
		Er stellt ein Analogon zu einem\\%
		\ \ \ \ \lstinline[language=C_var,basicstyle=\ttfamily]%
		{if(Bedingung)}\{\\%
		\ \ \ \ \ \ \ \}\lstinline[language=C_var,basicstyle=\ttfamily]%
		{else}\{\}\\%
		Konstrukt einer Programmiersprache dar.\\%
		Die untere Form des Entscheidungs Knoten stellt eine platzeffizientere
		Darstellungs-Variante dar.};%
%
%
%
%
%
%
%
%
%
	\node(mux)%
		[mux2=2,text width=5.4cm,%
		anchor=north,yshift=-\nodedistance-0.5cm]%
		at(decision2.south)%
		{Mehrfachauswahl};%
	\draw[arrow,<-](mux.north)--+(0,0.5cm);
%
	\node(muxwahl1)%
		[muxwahl2,%
		anchor=north,%
		xshift=-1.5cm]%
		at(mux.south)%
		{};%
	\node(muxwahl1_text)%
		[cleartext,anchor=center,inner sep=1em]%
		at(muxwahl1)%
		{Wahl\\1};%
	\node(muxwahl2)%
		[muxwahl2,%
		anchor=north,%
		xshift=1.5cm]%
		at(mux.south)%
		{};%
	\node(muxwahl2_text)%
		[cleartext,anchor=center,inner sep=1em]%
		at(muxwahl2)%
		{Wahl\\2};%
	\draw[arrow](muxwahl1.south)--+(0,-0.5cm);
	\draw[arrow](muxwahl2.south)--+(0,-0.5cm);
%
	\node(muxnote)%
		[legendnote,yshift=0.5cm]%
		at(mux.north east-|textleftanchor)%
		{Die Mehrfachauswahl ist eine Erweiterung des einfachen Entscheidungs
		Knotens. Sie erweitert die möglichen Ausgänge auf eine unbestimmte Anzahl.
		Statt einer Abfrage mit einem Ja-Nein-Ergebnis, kann ein Abfrage-Muster,
		ein Tupel aus mehreren Gleichheits-Prüfungen spezifiziert werden. Der
		Algorithmus folgt jenem Ausgang, dessen Wert mit dem Zustand der zu
		prüfenden Variable übereinstimmt. Das Analogon einer Programmiersprache
		wäre eines der beiden folgenden Konstrukte.\\%
		\begin{minipage}[t]{0.48\textwidth}%
		\noindent%
		\ \ \ \ \lstinline[language=C_var,basicstyle=\ttfamily]%
		{switch(Variable)}\{\nl%
		\ \ \ \ \lstinline[language=C_var,basicstyle=\ttfamily]%
		{case 0:}\nl%
		\ \ \ \ \ \ \ \ \lstinline[language=C_var,basicstyle=\ttfamily]%
		{break;}\nl%
		\ \ \ \ \lstinline[language=C_var,basicstyle=\ttfamily]%
		{case 1:}\nl%
		\ \ \ \ \ \ \ \ \lstinline[language=C_var,basicstyle=\ttfamily]%
		{break;}\nl%
		\ \ \ \ \}%
		\end{minipage}%
		\hspace{\fill}%
		\begin{minipage}[t]{0.48\textwidth}%
		\ \ \ \ \lstinline[language=C_var,basicstyle=\ttfamily]%
		{if(Bedingung1)}\{\nl%
		\ \ \ \ \}\lstinline[language=C_var,basicstyle=\ttfamily]%
		{else if(Bedingung2)}\{\nl%
		\ \ \ \ \}
		\end{minipage}%
		%
		};%
%
%
%
%
%
%
%
%
%
	\node(io)%
		[io,anchor=north,yshift=-\nodedistance-0.5cm]%
		at(mux.south|-muxnote.south)%
		{In / Out};%
	\draw[arrow](io.south)--+(0,-0.5cm);
	\draw[arrow,<-](io.north)--+(0,0.5cm);
%
	\node(ionote)%
		[legendnote,yshift=0.5cm]%
		at(io.north east-|textleftanchor)%
		{Schnittstellen, an denen der Algorithmus mit externen Entitäten
		unbestimmter Gestalt kommuniziert. Ein Beispiel bestünde in einer
		Netzwerkkommunikation via BSD Sockets zwischen zwei Hardware Plattformen,
		wobei der determinierte Algorithmus alleinig auf einem der Geräte
		ausgeführt wird.};%
%
%
%
%
%
%
%
%
%
	\node(interthread)%
		[interthread,anchor=north,yshift=-\nodedistance-0.5cm]%
		at($(io.south)+(0,-0.5cm)$)%
		{Inter Thread\\ Kommunikation};%
	\draw[arrow](interthread.south)--+(0,-0.5cm);
	\draw[arrow,<-](interthread.north)--+(0,0.5cm);
%
	\node(interthreadnote)%
		[legendnote,yshift=0.5cm]%
		at(interthread.north east-|textleftanchor)%
		{Ähnlich wie die I/O Knoten, mit dem Unterschied, dass diese Kommunikation
		innerhalb des Algorithmus stattfindet. Die Daten über diese Schnittstelle
		verlassen den Algorithmus nicht, sondern werden zwischen Teilen desselben
		Algorithmus ausgetauscht. Der Algorithmus kann über verschiedene
		Flussdiagramme verteilt sein: In etwa prinzipiell unabhängig agierende, jedoch
		untereinander interagierende Teile des Algorithmus. Kommunizieren diese
		Teile miteinander und würden bereits eigenständige Algorithmen darstellen,
		werden hingegen unter einem umfassenden Algorithmus subsummiert, erfolgt
		über Knoten dieser Gestalt die interne Kommunikation. Ein Beispiel für
		diese Art der Kommunikation wären die Threads innerhalb eines Programmes.};%
%
%
%
%
%
%
%
%
%
	\node(note)%
		[note,anchor=north east,xshift=-0.25cm,%
		yshift=-\nodedistance,yshift=-0.5cm]%
		at(interthread.south|-interthreadnote.south)%
		{Notiz};%
	\draw[arrow](note.south)--+(0,-0.5cm);
	\draw[arrow,<-](note.north)--+(0,0.5cm);
	\node(note2)%
		[note,anchor=north west,xshift=0.5cm]%
		at(note.north east)%
		{Notiz};%
%
	\node(notenote)%
		[legendnote,yshift=0.5cm]%
		at(note.north east-|textleftanchor)%
		{Eine Notiz, für Anmerkungen, die nicht direkt Teil der arbeitenden Routine
		eines Algorithmus sind. Sie vollziehen keinen Fortschritt oder stellen
		Verarbeitungsschritte dar. Sie dienen lediglich der besseren
		Verständlichkeit des Diagramms. Notizblöcke mögen als Teil des Algorithmus
		innerhalb des Fluss-Verlaufs vorkommen oder als generelle Anmerkung abseits
		vermerkt sein.\nl%
		Sie können z.B. Kommentare innerhalb eines Programm-Quelltextes repräsentieren
		oder besondere Stellen markieren.};%
%
%
%
%
%
%
%
%
%
%
%
%
%
%
%
%
%
%
%
%
%
%
%
%
\normalfont%
%
\end{tikzpicture}%
%
%
\DenKrTikzStandalonePost%